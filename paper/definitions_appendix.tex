\documentclass[12pt]{article}
\usepackage{mathptmx}
\usepackage[left=1in, right=1in, top=1in, bottom=1in]{geometry}
\usepackage{titling}
\usepackage{setspace}

\pretitle{\begin{center}\Huge\bfseries}
    \posttitle{\par\end{center}\vskip 0.5em}
    \preauthor{\begin{center}\Large\ttfamily}
    \postauthor{\end{center}}
    \predate{\par\large\centering}
    \postdate{\par}

\doublespacing
\begin{document}

\begin{center}
    \section*{DEFINITIONS APPENDIX}
\end{center}

\subsection*{Networks Concepts}

A network is understood as the the set of connections between
agents, or groups of agents, who share a common productive
activity. The agents groups will be named nodes, and the 
connection between the nodes will be named links. The links
represent that what is share between two nodes. Generally
speaking, the economic complexity literature used to 
compare exported goods types as nodes, and the value of those
exportations as links.

\subsection*{Networks in Complexity Theory}

On this paper, the research network uses the publications on
a specific field as node, and the cites that it has with other 
publications field as links. In the other hand, the industrial
networks use the production of specific kind of goods or
service as node, and the amount of other node as link with it.
A basic example of that usage is the Leontief input output
model.

\end{document}