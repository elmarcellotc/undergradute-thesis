\documentclass[12pt]{article}
\usepackage{mathptmx}
\usepackage[left=1in, right=1in, top=1in, bottom=1in]{geometry}
\usepackage{titling}
\usepackage{setspace}
\usepackage{booktabs}

\pretitle{\begin{center}\Huge\bfseries}
    \posttitle{\par\end{center}\vskip 0.5em}
    \preauthor{\begin{center}\Large\ttfamily}
    \postauthor{\end{center}}
    \predate{\par\large\centering}
    \postdate{\par}

\doublespacing
\begin{document}

\begin{center}
    \section*{RAW DATA STRUCTURE APPENDIX}
\end{center}

\subsection*{Scopus Data Table}

The main contribution of this working paper
is on the research data. The available 
research data of the Ecuador give us the
paper title, the authors, the year of
publication, and the university
affiliation. Using those variables, we
can build a dataset with the location
of the publication, and the available text.
The available text is necessary to classify
the paper as a industry based on the ISIC4
code. The following table is an example
of the raw publication dataset, we add
the ID variable as identifier of each paper\\


\begin{tabular}{llllll}
\toprule
ID & Year & Province & City & Authors & Text \\
\midrule
ECOTEC1 & 2018 & GUAYAS & GUAYAQUIL & Espinoza... & Ethi... \\
ECOTEC25 & 2018 & GUAYAS & GUAYAQUIL & Loor P... & Anal... \\
EPN23 & 2018 & PICHINCHA & QUITO & Silva M.... & Biopol... \\
EPN41 & 2018 & PICHINCHA & QUITO & Douillet... & Revisiti... \\
EPN61 & 2018 & PICHINCHA & QUITO & Sirunyan A.M... & Obser... \\
\bottomrule
\end{tabular}

\subsection*{International Standard Industrial Classification}

This is the common code used in economics
to classify an economic activity 

\subsection*{The Similarity Table}

The similarity say how similar is each paper
to each ISIC4 category. The similarity is a
value between 0 and 1, and this relationship
is got by SpaCy Python Library. The table 3
shows the first 10 values of the similarity

\begin{tabular}{lrrrr}
\toprule
ID & A0150 & C1074 & C1420 & C2021 \\
\midrule
ECOTEC0 & 0.317848 & 0.548013 & 0.446434 & 0.549180 \\
ECOTEC25 & 0.320490 & 0.610176 & 0.641187 & 0.616872 \\
EPN23 & 0.387861 & 0.613222 & 0.596166 & 0.638505 \\
EPN41 & 0.507463 & 0.829084 & 0.650273 & 0.792887 \\
EPN61 & 0.425886 & 0.675037 & 0.567544 & 0.686002 \\
\bottomrule
\end{tabular}

\subsection*{Publications Network Table}

The publication Network Table contains the information of
the publications interactions. It contains the publications
variables, an the identifier of the publication and 
the publication that replicates the other. 

It is the first part of the table, and represent the initial
node of the network because each variable contains the 
letter \textit{i} before the varaible name. The other part of
the table is, uses \textit{o} before the varaible name:

It is possible to have multiples authors from different
institutions, however, those observations will be different
and the iid to \textbf{iid}. The variable \textbf{pid} makes
a publication unique. Moreover, it is possible to see the
same pid for more than one iid in the table. At the
beginning of the table will be add an interaction id.
This makes the observation unique over the hole dataset.

The \textbf{code} variables refers to the industrial
classification of the paper. Those are the most similar
industry to the paper information, and the investigator 
can see the matching process on the \textit{PUBLICATIONS
DATA APPENDIX}.

The other variables are to show the location of the
institution, or to show the author of the publication. The
\textbf{iauthor:} The author of the publication. It could 
exists more than one iiauthor per ipid

\subsection*{Industrial Network}

The industrial network contains data from the ecuadorian 
service taxation institutions:
\textit{Servicio de Rentas Internas} (SRI). The observations
are aggregated by ecuadorian province, and the table shows
the year and the product ISIC4 code. As in the publications
table, the table divide the variables into \textit{i,o} as
input-output

\end{document}